\chapter{Entidades}

% User
\section{Usuário (User)}

\textbf{Descrição:} O usuário é o ator do sistema, ele será o sujeito que poderá gerenciadas a grande
maioria das entidades.

\textbf{Campos:} id, email, name, created\_at, updated\_at

\textbf{Relacionamento:} Um usuário tem uma agenda (e todos os seus eventos), uma credencial e vários rotinas, hábitos ou objetivos.

% Schedule
\section{Agenda (Schedule)}

\textbf{Descrição:} A agenda é responsável por registrar o cronograma e organizar os eventos em um calendário.

\textbf{Campos:} id, events

\textbf{Relacionamento:} Um usuário possuí uma agenda, e uma agenda possuí vários eventos

% ScheduleEvent
\section{\textit{Evento Agendado (ScheduleEvent)}}

\textbf{Descrição:} Evento Agendado é uma classe abstrata que representa qualquer alocação de tempo em uma agenda.

\textbf{Campos:} id, schedule\_id, title, description, status, start\_at, end\_at

\textbf{Relacionamento:} Um evento agendado pertence a uma agenda, podendo ser uma tarefa ou uma interação.

% Task
\section{Tarefa (Task)}

\textbf{Descrição:} Uma Tarefa é um tipo de evento agendado que irá acontecer apenas uma vez.

\textbf{Campos:} expire\_at

\textbf{Relacionamento:} Uma tarefa é um tipo de evento agendado, portanto pertence a uma agenda.

% Interaction
\section{Interação (Interaction)}

\textbf{Descrição:} Uma Interação é um tipo de evento agendado que registra as informações referentes a uma repetição de um ciclo.

\textbf{Campos:} cyclo\_id

\textbf{Relacionamento:} Uma tarefa é um tipo de evento agendado, portanto pertence a uma agenda. Ao mesmo tempo, uma interação pertence a um ciclo.

% Cyclo
\section{Ciclo (Cyclo)}

\textbf{Descrição:} Ciclo é um conjunto de repetição, ou seja, um conjunto de eventos agendados semelhantes. Os ciclos possuem tamanho e progresso mensurados em dias.

\textbf{Campos:} id, routine\_id, interactions, size, progress, period

\textbf{Relacionamento:} Um ciclo possuí várias interações, e pertence a uma rotina.

% Routine
\section{Rotina (Routine)}

\textbf{Descrição:} Uma rotina é a entidade que possuí ciclos, e eles são a métrica do progresso dessa entidade. Em uma rotina, toda repetição é importante ser realizada.

\textbf{Campos:} id, title, description, cyclos, status, progress, end\_at

\textbf{Relacionamento:} Uma rotina pode ter vários ciclos e pertence a um usuário.

% Objective
\section{Objetivo (Objetive)}

\textbf{Descrição:} Um objetivo é um tipo de rotina que possuí uma meta muito bem clara.

\textbf{Campos:} id, title, description, cyclos, status, progress, end\_at, progress, magnitude

\textbf{Relacionamento:} Um objetivo pode ter vários ciclos e pertence a um usuário.

% Habit
\section{Hábito (Habit)}

\textbf{Descrição:} Um hábito é um tipo de rotina onde a frequência de realização de cada repetição é muito importante.

\textbf{Campos:} id, title, description, cyclos, status, progress, end\_at

\textbf{Relacionamento:} Um objetivo pode ter vários ciclos e pertence a um usuário.
