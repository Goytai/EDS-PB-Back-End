\chapter{Definições}

\section{Tarefa:}

Uma tarefa é uma atividade a ser realizada em um dia, podendo ou não, ser
associada a um intervalo de tempo onde será cumprida. Caso não associada a um
intervalo de tempo, as tarefas podem se acumular para o próximo dia. Exemplo:

\begin{itemize}
  \item{Fazer o trabalho da faculdade}
  \item{Participar da reunião com os fornecedores - 13hrs até as 15hrs}
\end{itemize}


\section{Período:}

Nesse projeto, iremos considerar um período como o espaço de tempo em dias que
transcorre entre duas tarefas semelhantes.

\section{Rotina:}

Uma rotina, também pode ser chamado de tarefa periódica, é um conjunto de tarefas
semelhantes que se replicam após um \textbf{período} pré-definido. Exemplo:

\begin{itemize}
  \item{Ler o jornal diáriamente}
  \begin{itemize}
    \item{Dia 1: Ler o jornal}
    \item{Dia 2: Ler o jornal}
    \item{Dia n: Ler o jornal}
  \end{itemize}
\end{itemize}

\section{Repetição:}

Nesse projeto, iremos considerar uma repetição como a unidade de uma tarefa que
compõe uma \textbf{rotina}.

\newpage

\section{Objetivo:}

Um objetivo é uma rotina que possuí uma meta quantitativa. Nesse caso, ao concluir
cada \textbf{repetição}, deve-se informar o avanço realizado. Exemplo:

\begin{itemize}
  \item{Ler o livro Código Limpo - 425 Páginas}
  \begin{itemize}
    \item{Dia 1: Ler o livro Código Limpo - Foi lido +10 páginas}
    \item{Dia 2: Ler o livro Código Limpo - Foi lido +30 páginas}
    \item{Dia n: Ler o livro Código Limpo - Foi lido +100 páginas}
  \end{itemize}
\end{itemize}

\section{Ciclo:}

Nesse projeto, um ciclo são 22 dias consecutivos de repetições sendo realizadas.
A tolerância para repetições não realizadas, deve depender da quantidade de ciclos
já atingidos.

\section{Hábito:}

Um hábito é composto por vários objetivos ou rotinas. Esses hábitos devem ser
mensurados em dias e categorizados em ciclos. Exemplo:

\begin{itemize}
  \item{Hábito de ler:}
  \begin{itemize}
    \item{Ler o jornal diáriamente \textbf{(rotina)}}
    \begin{itemize}
      \item{Dia 1: Ler o jornal}
      \item{Dia 2: Ler o jornal}
      \item{Dia n: Ler o jornal}
    \end{itemize}
    \item{Ler o livro Código Limpo - 425 Páginas \textbf{(objetivo)}}
    \begin{itemize}
      \item{Dia 1: Ler o livro Código Limpo - Foi lido +10 páginas}
      \item{Dia 2: Ler o livro Código Limpo - Foi lido +30 páginas}
      \item{Dia n: Ler o livro Código Limpo - Foi lido +100 páginas}
    \end{itemize}

  \end{itemize}
\end{itemize}

\section{Programação:}

Nesse projeto, uma programação se refere a qualquer tipo de evento que esteja
programado. Ou seja, qualquer tarefa, rotina, repetição, objetivo ou hábito.

