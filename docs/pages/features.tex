\chapter{Principais funcionalidades}

\begin{description}

  \item[1. Interface amigável:] O sistema deve possuir uma interface de usuário web e
    mobile, que facilite a utilização e navegação.

  \item[2. Autenticação:] O cliente deve ser capaz de se cadastrar e autenticar no
    sistema, fornecendo informações básicas, como nome, e-mail e senha.

  \item[3. Gestão de tarefas:] O cliente deve conseguir administrar uma tarefa
    em sua conta. Podendo atribuir, ou não, um intervalo de tempo a essa tarefa e
    também definir se a tarefa irá se acumular para o próximo dia, caso não
    realizada.

  \item[4. Gestão de rotinas:] O cliente deve ser capaz de administrar as rotinas
    de sua conta. Definindo o período de cada repetição, e a tarefa que irá ser repetida.
    Para que isso seja possível, deve-se acrescentar ao sistema de gestão de tarefas,
    a estrutura para uma tarefa periódica.

  \item[5. Gestão de objetivos:] O cliente deve ser capaz de administrar os objetivos
    de sua conta. Definindo o período de cada repetição, a tarefa que irá ser repetida
    e as metas que irá ser alcançada. Para que isso seja possível, deve-se acrescentar
    ao sistema de gestão de tarefas, o registro quantitativo do progresso.

  \item[6. Sistema de progresso:] O sistema deve ser capaz de mostrar o progresso
    dos objetivos de forma quantitativa, enquanto as rotinas e os hábitos devem ter
    o progresso exibido utilizando ciclo atual e o avanço no ciclo.

  \item[7. Gestão de hábitos:] O cliente deve ser capaz de administrar os hábitos
    em sua conta, definir as rotinas ou objetivos que serão repetidos. Também será
    necessário adicionar ao sistema de progresso, a possibilidade de reiniciar o
    avanço caso a tolerância de repetições do ciclo seja excedida.

  \item[8. Sistema de aviso:] O cliente deve ser capaz de configurar quais
  tarefas, rotinas ou hábitos. Que vão acionar uma notificação/alerta quando
  estiver próximo ao horário planejado para a o evento.

  \item[9. Suporte ao cliente:] O sistema deve oferecer suporte ao cliente, com
  opções de contato para dúvidas, problemas ou qualquer assistência necessária.

  \item[10. Integração:] O cliente deve conseguir sincronizar suas programações com
  serviços externos, como Google Calendar e/ou com outros serviços do gênero.

\end{description}
